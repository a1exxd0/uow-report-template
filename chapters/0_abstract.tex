\begin{abstract}
    Interactive proofs for quantum learning represent a promising frontier for enabling secure delegation of quantum machine learning tasks to untrusted quantum servers. However, practical deployment of these protocols faces significant challenges when operating under realistic resource constraints, including limited quantum memory, finite coherence times, and restricted communication bandwidth. This project explores the development and optimization of resource-efficient interactive verification protocols that can operate effectively within the constraints of near-term quantum devices. The first objective is to design and analyze modified interactive proof systems that minimize quantum resource requirements while maintaining security guarantees, focusing on protocols that can accommodate noisy hardware limitations and decoherence effects. The second goal involves developing protocols that can dynamically adapt to varying decoherence levels in available quantum hardware, potentially adjusting verification strategies based on theoretical assessment of device coherence properties. This work aims to bridge the gap between the theoretical promise of quantum learning verification and its practical applicability on resource-constrained quantum systems, potentially enabling broader adoption of verified quantum machine learning in real-world applications.
\end{abstract}
